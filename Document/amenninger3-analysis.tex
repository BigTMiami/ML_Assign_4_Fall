\def\year{2021}\relax
%File: formatting-instructions-latex-2021.tex
%release 2021.1
\documentclass[letterpaper]{article} % DO NOT CHANGE THIS
\usepackage{aaai21}  % DO NOT CHANGE THIS
\usepackage{times}  % DO NOT CHANGE THIS
\usepackage{helvet} % DO NOT CHANGE THIS
\usepackage{courier}  % DO NOT CHANGE THIS
\usepackage[hyphens]{url}  % DO NOT CHANGE THIS
\usepackage{graphicx} % DO NOT CHANGE THIS

% ADJUSTED BY AFM ----------------------------------------------------------
\usepackage{amsmath} % ENTERED BY AFM
\usepackage{subcaption}
\usepackage[table,usenames,dvipsnames]{xcolor}



\urlstyle{rm} % DO NOT CHANGE THIS
\def\UrlFont{\rm}  % DO NOT CHANGE THIS
\usepackage{natbib}  % DO NOT CHANGE THIS AND DO NOT ADD ANY OPTIONS TO IT
\usepackage{caption} % DO NOT CHANGE THIS AND DO NOT ADD ANY OPTIONS TO IT
\frenchspacing  % DO NOT CHANGE THIS
\setlength{\pdfpagewidth}{8.5in}  % DO NOT CHANGE THIS
\setlength{\pdfpageheight}{11in}  % DO NOT CHANGE THIS
\nocopyright
%PDF Info Is REQUIRED.
% For /Author, add all authors within the parentheses, separated by commas. No accents or commands.
% For /Title, add Title in Mixed Case. No accents or commands. Retain the parentheses.
\pdfinfo{
/Title (ML Assignment 4)
/Author (Anthony Menninger)
/TemplateVersion (2021.1)
} %Leave this
% /Title ()
% Put your actual complete title (no codes, scripts, shortcuts, or LaTeX commands) within the parentheses in mixed case
% Leave the space between \Title and the beginning parenthesis alone
% /Author ()
% Put your actual complete list of authors (no codes, scripts, shortcuts, or LaTeX commands) within the parentheses in mixed case.
% Each author should be only by a comma. If the name contains accents, remove them. If there are any LaTeX commands,
% remove them.

% DISALLOWED PACKAGES
% \usepackage{authblk} -- This package is specifically forbidden
% \usepackage{balance} -- This package is specifically forbidden
% \usepackage{color (if used in text)
% \usepackage{CJK} -- This package is specifically forbidden
% \usepackage{float} -- This package is specifically forbidden
% \usepackage{flushend} -- This package is specifically forbidden
% \usepackage{fontenc} -- This package is specifically forbidden
% \usepackage{fullpage} -- This package is specifically forbidden
% \usepackage{geometry} -- This package is specifically forbidden
% \usepackage{grffile} -- This package is specifically forbidden
% \usepackage{hyperref} -- This package is specifically forbidden
% \usepackage{navigator} -- This package is specifically forbidden
% (or any other package that embeds links such as navigator or hyperref)
% \indentfirst} -- This package is specifically forbidden
% \layout} -- This package is specifically forbidden
% \multicol} -- This package is specifically forbidden
% \nameref} -- This package is specifically forbidden
% \usepackage{savetrees} -- This package is specifically forbidden
% \usepackage{setspace} -- This package is specifically forbidden
% \usepackage{stfloats} -- This package is specifically forbidden
% \usepackage{tabu} -- This package is specifically forbidden
% \usepackage{titlesec} -- This package is specifically forbidden
% \usepackage{tocbibind} -- This package is specifically forbidden
% \usepackage{ulem} -- This package is specifically forbidden
% \usepackage{wrapfig} -- This package is specifically forbidden
% DISALLOWED COMMANDS
% \nocopyright -- Your paper will not be published if you use this command
% \addtolength -- This command may not be used
% \balance -- This command may not be used
% \baselinestretch -- Your paper will not be published if you use this command
% \clearpage -- No page breaks of any kind may be used for the final version of your paper
% \columnsep -- This command may not be used
% \newpage -- No page breaks of any kind may be used for the final version of your paper
% \pagebreak -- No page breaks of any kind may be used for the final version of your paperr
% \pagestyle -- This command may not be used
% \tiny -- This is not an acceptable font size.
% \vspace{- -- No negative value may be used in proximity of a caption, figure, table, section, subsection, subsubsection, or reference
% \vskip{- -- No negative value may be used to alter spacing above or below a caption, figure, table, section, subsection, subsubsection, or reference

\setcounter{secnumdepth}{0} %May be changed to 1 or 2 if section numbers are desired.

% The file aaai21.sty is the style file for AAAI Press
% proceedings, working notes, and technical reports.
%

% Title

% Your title must be in mixed case, not sentence case.
% That means all verbs (including short verbs like be, is, using,and go),
% nouns, adverbs, adjectives should be capitalized, including both words in hyphenated terms, while
% articles, conjunctions, and prepositions are lower case unless they
% directly follow a colon or long dash


%Example, Single Author, ->> remove \iffalse,\fi and place them surrounding AAAI title to use it
\title{
Machine Learning - CS 7641
Assignment 4
	
}
\author {
    % Author
    Anthony Menninger \\
}

\affiliations{
    Georgia Tech OMSCS Program \\
    amenninger3\\
    tmenninger@gatech.edu

}

\begin{document}

\maketitle

\begin{abstract}
This paper explores Markov Decision Processes through the use of Value Iteration, Policy Iteration and Q Learning.  It looks at a simple "Forest" MDP and a larger Frozen Lake "Grid World".
\end{abstract}

\section{Problem Introduction}

A key element in both Markov Decision Processes (MDPs) is the idea of a discount, which is often referred to as \textbf{\emph{Gamma ($\gamma$)}}.  This discount factor, between 1 and 0, allows solutions for infinite MDPs by discounting future value by \textbf{\emph{Gamma}} for each step into the future.  In an infinite activity, this ends up valuing future value at $ \frac{1}{1- \gamma}$.  Gamma close to 0 creates a very close horizon where only immediate rewards are maximized, while Gamma close to 1 creates a long horizon, where maximizing long term reward is emphasized.  As will be seen, MDPs are very sensitive to this and small changes can create very different solutions.

The key toolset I used was the hiive fork [3] of the MDP Toolbox for Python [2], which is derived from the MDP Toolbox [1].  I also used the OpenAI Gym Frozen Lake environment [4] for the larger Frozen Lake MDP.

\subsection{Forest MDP}
\textbf{\emph{Forest MDP}} is a simple MDP from the hiive MDP Toolbox [2] that models the value of a forest with respect to two actions that can be performed each year: \textbf{\emph{wait}} or \textbf{\emph{cut}}.  There is a stochastic element of forest fires that occur with probability \textbf{\emph{p}}.  Each year is a state, with a max of \textbf{\emph{S}} years / states.  Cutting deterministically transitions to the initial state,  \textbf{\emph{state=0}} and provides a \textbf{\emph{cutting reward}}.  Waiting transitions to the next year, or if in the max state, remains there.  Forest fires provide a stochastic element, with a fire transitioning back to the initial state.  A \textbf{\emph{waiting reward}} only occurs in the max state, \textbf{\emph{state=S}}.  This is a continual MDP, with no terminal or absorbing states. 

The base setup for \textbf{\emph{Forest}} is seven years (\textbf{\emph{S=7}}), a 10\% chance of forest fire (\textbf{\emph{p=0.1}}), a cutting reward of 2 (\textbf{\emph{cutting reward=2}}) and a waiting reward of 4 (\textbf{\emph{waiting reward = 4}}).

In the context of MDP's, each state has an action that maximizes expected value and the set of maximizing actions for all states is called a \textbf{\emph{policy}}.  This policy can be examined to determine what rational actors are likely going to do. 

There are several very interesting aspects to this problem.  \textbf{\emph{Forest MDP}} models a key choice being made today around ecology and the environment.  What are the rewards needed to keep forests around?  How do maximizing actions (\textbf{\emph{policy}}) change as the chance of forest fires increase?  How does the length of the considered horizon (\textbf{\emph{Gamma discount}}) influence expected outcomes.  

This also highlights a key challenge in MDP's and Reinforcement Learning, which is understanding rewards and setting them appropriately.  \textbf{\emph{Forest MDP}}  allows modeling of different situations to inform the construction of public policy for governments, NGO's and corporations.  The \textbf{\emph{cutting reward}} might be clearly modeled using expected timber value, but other types of reward may want to be considered in the \textbf{\emph{waiting reward}}, such as carbon reduction and maintenance of ecological diversity.  This MDP can be used to model economic rewards and penalties to compel desired outcomes, such as a carbon tax that would reduce the value of the \textbf{\emph{cutting reward}}.

As described, this MDP is more likely to be "solved" using Value Iteration (VI) or Policy Iteration (PI) in order to use it for comparing different tradeoffs and outcomes.

\subsection{Lake MDP}
\textbf{\emph{Lake MDP}} is a larger MDP that uses the OpenAI gym Frozen Lake [4,5] to create its environment.   It is a 2D grid with four actions to move \textbf{\emph{up, down, left, right}} and can be of \textbf{\emph{Size}}, which is the number of grids on one size of the square, creating \textbf{\emph{Size x Size}} states.  The lake can be set to not slippery, where the action is deterministic, or slippery, where the action is stochastic.  When slippery, the action has a 1/3 chance of moving in the desired direction, a 1/3 to the left of the desired direction and 1/3 to the right of the desired direction, which is fairly stochastic.  This is an episodic MDP where the player starts in the first state - (0,0) upper left corner.  The goal state (size, size) is in the bottom right corner and has a reward of 1.  There are also random holes which are absorbing states with a reward of 0.

I created Small (\textbf{\emph{Size=4}}), Medium (\textbf{\emph{Size=8}}), and Large (\textbf{\emph{Size=16}}) worlds to explore the impacts of different size worlds.  

This MDP is interesting because it embodies a core Reinforcement Learning challenge of exploring an environment and learning what to do.  It describes a common, everyday task of exploring an environment and learning where the rewards and traps are.  Robots are often in a situation where they need to move about and find out what they should be doing.  In addition, it creates a much more interesting set of transitions between the different states, as compared with the \textbf{\emph{Forest MDP}}, where transitions are either next state or back to beginning.  Also, \textbf{\emph{Lake MDP}} is episodic vs continual for \textbf{\emph{Forest MDP}}.

As described, this MDP is more likely to be "learned" in a setting using Q Learning to explore how agents work.

\section{Forest and Lake MDPs Value Iteration and Policy Iteration}

\section{Lake MDP Value Iteration and Policy Iteration}

\section{Forest MDP Q Learning}

\section{Lake MDP Q Learning}

\section{Summary}

\section{References}
\begin{tabular}{l p{2.75in}}
\\
1 & Markov Decision Processes (MDP) Toolbox, https://miat.inrae.fr/MDPtoolbox/ Accessed: 11/1/2022.
\\
2 & Markov Decision Process (MDP) Toolbox for Python, https://github.com/sawcordwell/pymdptoolbox: Accessed: 11/1/2022.
\\
3 & hiive Fork: Markov Decision Process (MDP) Toolbox for Python, https://github.com/hiive/hiivemdptoolbox: Accessed: 11/1/2022.
\\
4 & Brockman, G. et al., 2016. Openai gym.
\\
5 & Openai gym Frozen Lake Documentation,  https://www.gymlibrary.dev/environments/toy\_text/frozen\_lake/: Accessed: 11/1/2022.

\end{tabular}
\end{document}
